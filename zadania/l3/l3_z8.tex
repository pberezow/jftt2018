\documentclass{article}
\title{Lista 3, zadanie 8}
\author{Piotr Berezowski, 236749}

\usepackage{polski}
\usepackage[utf8]{inputenc}
\usepackage{enumerate}
\usepackage{amssymb}

\begin{document}
	\maketitle
	\pagenumbering{gobble}
	\newpage
	\pagenumbering{arabic}
	
	\section{Treść zadania}

    Znaleźć DFA o minimalnej liczbie stanów równoważny automatowi

    $$ M = (\{a, b, c, d, e, f, g, h\}, \{0, 1\}, \delta, a, \{d\}), $$

    gdzie $\delta$ ma postać 

    \begin{table}[h!]
		\begin{center}
			\begin{tabular}{c||c|c}
				& \textbf{0} & \textbf{1} \\
                \hline
                \hline
                \textbf{a} & b & a \\
                \hline
                \textbf{b} & a & c \\
                \hline
                \textbf{c} & d & b \\
                \hline
                \textbf{d} & d & a \\
                \hline
                \textbf{e} & d & f \\
                \hline
                \textbf{f} & g & e \\
                \hline
                \textbf{g} & f & g \\
                \hline
                \textbf{h} & g & d \\
			\end{tabular}
		\end{center}
	\end{table}

    \section{Rozwiązanie}

    Automat $M$ to piątka postaci $(Q, \Sigma, \delta, q_0, F)$, gdzie:
    
    \begin{enumerate}
        \item $Q$ - skończony zbiór stanów
        \item $\Sigma$ - skończony alfabet
        \item $\delta$ - funkcja przejścia $Q \times \Sigma \rightarrow Q$
        \item $q_0$ - stan początkowy
        \item $F$ - zbiór stanów akceptujących
    \end{enumerate}

    Rozwiązanie polega na zastosowaniu algorytmu minimalizacji. Kroki algorytmu:

    \begin{flushleft}
        $\textbf{forall } p \in F \wedge q \in Q \setminus F \textbf{ do:}$\break
        $|\quad \textnormal{oznacz parę } (p, q)$\break
        $\textbf{endfor}$\break
        $\textbf{forall } p, q \in (F \times F) \cup (Q \setminus F \times Q \setminus F) \textbf{ do:}$\break
        $|\quad \textbf{if } \exists_{a \in \Sigma} \textnormal{ t. że } (\delta(p, a), \delta(q, a)) \textnormal{ jest oznaczona } \textbf{ then:}$\break
        $|\quad |\quad \textnormal{oznacz }(p, q)$\break
        $|\quad |\quad \textnormal{oznacz rekurencyjnie wszystkie nieoznaczone pary na liście } (p, q)$\break
        $|\quad \textbf{else:}$\break
        $|\quad |\quad \textbf{forall } a \in \Sigma \textbf{ do:}$\break
        $|\quad |\quad |\quad \textnormal{umieść parę } (p, q) \textnormal{ na liście } (\delta(p, a), \delta(q, a)), \textnormal{ o ile } \delta(p, a) \neq \delta(q, a)$\break
        $|\quad |\quad \textbf{endfor}$\break
        $|\quad \textbf{endif}$\break
        $\textbf{endfor}$\break
    \end{flushleft}

    Zaczynamy od zapisania tablicy zawierającej jedną pozycję dla każdej pary stanów.

    \begin{table}[h!]
        \begin{tabular}{cccccccc}
        b                      & c                     & d                     & e                     & f                     & g                     & h                     &   \\ \cline{1-7}
        \multicolumn{1}{|c|}{} & \multicolumn{1}{c|}{} & \multicolumn{1}{c|}{} & \multicolumn{1}{c|}{} & \multicolumn{1}{c|}{} & \multicolumn{1}{c|}{} & \multicolumn{1}{c|}{} & a \\ \cline{1-7}
        \multicolumn{1}{c|}{}  & \multicolumn{1}{c|}{} & \multicolumn{1}{c|}{} & \multicolumn{1}{c|}{} & \multicolumn{1}{c|}{} & \multicolumn{1}{c|}{} & \multicolumn{1}{c|}{} & b \\ \cline{2-7}
                               & \multicolumn{1}{c|}{} & \multicolumn{1}{c|}{} & \multicolumn{1}{c|}{} & \multicolumn{1}{c|}{} & \multicolumn{1}{c|}{} & \multicolumn{1}{c|}{} & c \\ \cline{3-7}
                               &                       & \multicolumn{1}{c|}{} & \multicolumn{1}{c|}{} & \multicolumn{1}{c|}{} & \multicolumn{1}{c|}{} & \multicolumn{1}{c|}{} & d \\ \cline{4-7}
                               &                       &                       & \multicolumn{1}{c|}{} & \multicolumn{1}{c|}{} & \multicolumn{1}{c|}{} & \multicolumn{1}{c|}{} & e \\ \cline{5-7}
                               &                       &                       &                       & \multicolumn{1}{c|}{} & \multicolumn{1}{c|}{} & \multicolumn{1}{c|}{} & f \\ \cline{6-7}
                               &                       &                       &                       &                       & \multicolumn{1}{c|}{} & \multicolumn{1}{c|}{} & g \\ \cline{7-7}
        \end{tabular}
    \end{table}

    Następnie oznaczamy pary $(p, q)$ takie, że $p \in F \wedge q \in Q \setminus F$.

    \begin{table}[h!]
        \begin{tabular}{cccccccc}
        b                      & c                     & d                      & e                      & f                      & g                      & h                      &   \\ \cline{1-7}
        \multicolumn{1}{|c|}{} & \multicolumn{1}{c|}{} & \multicolumn{1}{c|}{X} & \multicolumn{1}{c|}{}  & \multicolumn{1}{c|}{}  & \multicolumn{1}{c|}{}  & \multicolumn{1}{c|}{}  & a \\ \cline{1-7}
        \multicolumn{1}{c|}{}  & \multicolumn{1}{c|}{} & \multicolumn{1}{c|}{X} & \multicolumn{1}{c|}{}  & \multicolumn{1}{c|}{}  & \multicolumn{1}{c|}{}  & \multicolumn{1}{c|}{}  & b \\ \cline{2-7}
                               & \multicolumn{1}{c|}{} & \multicolumn{1}{c|}{X} & \multicolumn{1}{c|}{}  & \multicolumn{1}{c|}{}  & \multicolumn{1}{c|}{}  & \multicolumn{1}{c|}{}  & c \\ \cline{3-7}
                               &                       & \multicolumn{1}{c|}{}  & \multicolumn{1}{c|}{X} & \multicolumn{1}{c|}{X} & \multicolumn{1}{c|}{X} & \multicolumn{1}{c|}{X} & d \\ \cline{4-7}
                               &                       &                        & \multicolumn{1}{c|}{}  & \multicolumn{1}{c|}{}  & \multicolumn{1}{c|}{}  & \multicolumn{1}{c|}{}  & e \\ \cline{5-7}
                               &                       &                        &                        & \multicolumn{1}{c|}{}  & \multicolumn{1}{c|}{}  & \multicolumn{1}{c|}{}  & f \\ \cline{6-7}
                               &                       &                        &                        &                        & \multicolumn{1}{c|}{}  & \multicolumn{1}{c|}{}  & g \\ \cline{7-7}
        \end{tabular}
    \end{table}

    Wykonując drugi krok algorytmu oznaczamy kolejne pary.

    \begin{enumerate}
        \item $(a,b) \rightarrow^0 (b, a)$ - para $(b, a)$ nie jest oznaczona, więc sprawdzamy dalej.
        \item $(a,b) \rightarrow^1 (a, c)$ - para $(a, c)$ nie jest oznaczona, więc sprawdzamy dalej.
        \item $(a,c) \rightarrow^0 (b, d)$ - para $(b, d)$ jest oznaczona, więc oznaczamy $(a, c)$. Następnie sprawdzamy czy wcześniej 
            po prawej stronie nie pojawiła się gdzieś nowo oznaczona para $(a, c)$, Jeśli tak, to oznaczamy lewą stronę i rekurencyjnie 
            wykonujemy sprawdzenie dla nowej pary(lewej strony). W tym wypadku wcześniej było $(a,b) \rightarrow^1 (a, c)$, więc 
            oznaczamy $(a, b)$ i sprawdzamy czy $(a, b)$ nie pojawiło się wcześniej po prawej stronie. Nie pojawiło się, więc możemy iść dalej.
        \item $(a,e) \rightarrow^0 (b, d)$ - para $(b, d)$ jest oznaczona, więz oznaczamy $(a, e)$.
        \item $(a,f) \rightarrow^0 (b, g)$ - para $(b, g)$ nie jest oznaczona, więc sprawdzamy dalej.
        \item $(a,f) \rightarrow^1 (a, e)$ - para $(a, e)$ jest oznaczona, więc oznaczamy $(a, f)$.
        \item $(a,g) \rightarrow^0 (b, f)$ - para $(b, f)$ nie jest oznaczona, więc sprawdzamy dalej.
        \item $(a,g) \rightarrow^1 (a, g)$ - para $(a, g)$ nie jest oznaczona, więc sprawdzamy dalej.
        \item $(a,h) \rightarrow^0 (b, g)$ - para $(b, g)$ nie jest oznaczona, więc sprawdzamy dalej.
        \item $(a,h) \rightarrow^1 (a, d)$ - para $(a, d)$ jest oznaczona, więc oznaczamy $(a, h)$.
        \item $(b,c) \rightarrow^0 (a, d)$ - para $(a, d)$ jest oznaczona, więc oznaczamy $(b, c)$.
        \item $(b,e) \rightarrow^0 (a, d)$ - para $(a, d)$ jest oznaczona, więc oznaczamy $(b, e)$.
        \item $(b,f) \rightarrow^0 (a, g)$ - para $(a, g)$ nie jest oznaczona, więc sprawdzamy dalej.
        \item $(b,f) \rightarrow^1 (c, e)$ - para $(c, e)$ nie jest oznaczona, więc sprawdzamy dalej.
        \item $(b,g) \rightarrow^0 (a, f)$ - para $(a, f)$ jest oznaczona, więc oznaczamy $(b, g)$.
        \item $(b,h) \rightarrow^0 (a, g)$ - para $(a, g)$ nie jest oznaczona, więc sprawdzamy dalej.
        \item $(b,h) \rightarrow^1 (c, d)$ - para $(c, d)$ jest oznaczona, więc oznaczamy $(b, h)$.
        \item $(c,e) \rightarrow^0 (d, d)$ - para $(d, d)$ nie jest oznaczona, więc sprawdzamy dalej.
        \item $(c,e) \rightarrow^1 (b, f)$ - para $(b, f)$ nie jest oznaczona, więc sprawdzamy dalej.
        \item $(c,f) \rightarrow^0 (d, g)$ - para $(d, g)$ jest oznaczona, więc oznaczamy $(c, f)$.
        \item $(c,g) \rightarrow^0 (d, f)$ - para $(d, f)$ jest oznaczona, więc oznaczamy $(c, g)$.
        \item $(c,h) \rightarrow^0 (d, g)$ - para $(d, g)$ jest oznaczona, więc oznaczamy $(c, h)$.
        \item $(e,f) \rightarrow^0 (d, g)$ - para $(d, g)$ jest oznaczona, więc oznaczamy $(e, f)$.
        \item $(e,g) \rightarrow^0 (d, f)$ - para $(d, f)$ jest oznaczona, więc oznaczamy $(e, g)$.
        \item $(e,h) \rightarrow^0 (d, g)$ - para $(d, g)$ jest oznaczona, więc oznaczamy $(e, h)$.
        \item $(f,g) \rightarrow^0 (g, f)$ - para $(g, f)$ nie jest oznaczona, więc sprawdzamy dalej.
        \item $(f,g) \rightarrow^1 (e, g)$ - para $(e, g)$ jest oznaczona, więc oznaczamy $(f, g)$.
        \item $(f,h) \rightarrow^0 (g, g)$ - para $(g, g)$ nie jest oznaczona, więc sprawdzamy dalej.
        \item $(f,h) \rightarrow^1 (e, d)$ - para $(e, d)$ jest oznaczona, więc oznaczamy $(f, h)$.
        \item $(g,h) \rightarrow^0 (f, g)$ - para $(f, g)$ jest oznaczona, więc oznaczamy $(g, h)$.
    \end{enumerate}
    
    Ostatecznie tabela wygląda w ten sposób:

    \begin{table}[h!]
        \begin{tabular}{cccccccc}
        b                       & c                      & d                      & e                      & f                      & g                      & h                      &   \\ \cline{1-7}
        \multicolumn{1}{|c|}{X} & \multicolumn{1}{c|}{X} & \multicolumn{1}{c|}{X} & \multicolumn{1}{c|}{X} & \multicolumn{1}{c|}{X} & \multicolumn{1}{c|}{}  & \multicolumn{1}{c|}{X} & a \\ \cline{1-7}
        \multicolumn{1}{c|}{}   & \multicolumn{1}{c|}{X} & \multicolumn{1}{c|}{X} & \multicolumn{1}{c|}{X} & \multicolumn{1}{c|}{}  & \multicolumn{1}{c|}{X} & \multicolumn{1}{c|}{X} & b \\ \cline{2-7}
                                & \multicolumn{1}{c|}{}  & \multicolumn{1}{c|}{X} & \multicolumn{1}{c|}{}  & \multicolumn{1}{c|}{X} & \multicolumn{1}{c|}{X} & \multicolumn{1}{c|}{X} & c \\ \cline{3-7}
                                &                        & \multicolumn{1}{c|}{}  & \multicolumn{1}{c|}{X} & \multicolumn{1}{c|}{X} & \multicolumn{1}{c|}{X} & \multicolumn{1}{c|}{X} & d \\ \cline{4-7}
                                &                        &                        & \multicolumn{1}{c|}{}  & \multicolumn{1}{c|}{X} & \multicolumn{1}{c|}{X} & \multicolumn{1}{c|}{X} & e \\ \cline{5-7}
                                &                        &                        &                        & \multicolumn{1}{c|}{}  & \multicolumn{1}{c|}{X} & \multicolumn{1}{c|}{X} & f \\ \cline{6-7}
                                &                        &                        &                        &                        & \multicolumn{1}{c|}{}  & \multicolumn{1}{c|}{X} & g \\ \cline{7-7}
        \end{tabular}
    \end{table}
    \break
    Stany $a$ i $g$, $b$ i $f$, $c$ i $e$ są parami równoważne.\hfill\break
    Stan $h$ nie jest osiągalny ze stanu początkowego, więc możemy go usunąć.


\end{document}