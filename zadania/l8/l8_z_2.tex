\documentclass{article}
\title{Lista 8, zadanie 2}
\author{Piotr Berezowski, 236749}

\usepackage{polski}
\usepackage[utf8]{inputenc}
\usepackage{enumerate}
\usepackage{amssymb}

\begin{document}
	\maketitle
	\pagenumbering{gobble}
	\newpage
	\pagenumbering{arabic}
	
	\section{Treść zadania}

    Jak wyglądałaby tablica priorytetów obliczona algorytmem dla gramatyki

    $$ E \rightarrow E + E | E - E | E * E | E / E | id $$

    \section{Rozwiązanie}

    Tablicę priorytetów wypełniamy według następujących regół:

    Dla każdej produkcji $A \rightarrow x_1 \dots x_n$
    \begin{enumerate}
        \item Jeśli $x_i$ i $x_{i+1}$ są terminalami, to $x_i \doteq x_{i+1}$
        \item Jeśli $x_i$ i $x_{i+2}$ są terminalami, a $x_{i+1}$ jest nieterminalem, to $x_i \doteq x_{i+2}$
        \item Jeśli $x_i$ jest terminalem, a $x_{i+1}$ jest nieterminalem, to dla każdego $a \in LEADING(x_{i+1})$ mamy $x_i \lessdot a$
        \item Jeśli $x_i$ jest nieterminalem, a $x_{i+1}$ jest terminalem, to dla każdego $a \in TRAILING(x_i)$ mamy $a \gtrdot x_{i+1}$
    \end{enumerate} 

    Żeby wyznaczyć tablicę priorytetów dla podanej gramatyki, zacznijmy od wyznaczenia zbiorów $LEADING$ i $TRAILING$.

    $$ LEADING(E) = \{id, +, -, *, /\} $$
    $$ TRAILING(E) = \{id, +, -, *, /\} $$

    Dla podanej gramatyki występują konflikty wynikające z niejednoznacznego drzewa wyprowadzeń dla napisów. To znaczy, że napisy możemy wyprowadzić na 
    kilka różnych sposobów.
    Weźmy na przykład:
    
    $$id * id + id$$
    
    Powyższy napis możemy wyprowadzić na dwa różne sposoby:

    $$E \Rightarrow E * E \Rightarrow id * E \Rightarrow id * E + E \Rightarrow id * id + E \Rightarrow id * id + id$$
    $$E \Rightarrow E + E \Rightarrow E * E + E \Rightarrow id * E + E \Rightarrow id * id + E \Rightarrow id * id + id$$

    Konflikty są spowodowane tym że użycie regół 3 i 4 nadpisuje komórki w tablicy priorytetów, która wyznaczona według regół wygląda w następujący sposób:

	\begin{table}[h!]
		\begin{center}
			\caption{Tabela priorytetów}
			\label{tab:table1}
			\begin{tabular}{c||c|c|c|c|c|c}
				& \textbf{+} & \textbf{-} & \textbf{*} & \textbf{/} & \textbf{id} & \textbf{\$}\\
                \hline
                \hline
                \textbf{+} & $\lessdot \gtrdot$ & $\lessdot \gtrdot$ & $\lessdot \gtrdot$ & $\lessdot \gtrdot$ & $\lessdot \gtrdot$ & $\gtrdot$ \\
                \hline
                \textbf{-} & $\lessdot \gtrdot$ & $\lessdot \gtrdot$ & $\lessdot \gtrdot$ & $\lessdot \gtrdot$ & $\lessdot \gtrdot$ & $\gtrdot$ \\
                \hline
                \textbf{*} & $\lessdot \gtrdot$ & $\lessdot \gtrdot$ & $\lessdot \gtrdot$ & $\lessdot \gtrdot$ & $\lessdot \gtrdot$ & $\gtrdot$ \\
                \hline
                \textbf{/} & $\lessdot \gtrdot$ & $\lessdot \gtrdot$ & $\lessdot \gtrdot$ & $\lessdot \gtrdot$ & $\lessdot \gtrdot$ & $\gtrdot$ \\
                \hline
                \textbf{id} & $\lessdot \gtrdot$ & $\lessdot \gtrdot$ & $\lessdot \gtrdot$ & $\lessdot \gtrdot$ &  & $\gtrdot$ \\
                \hline
                \textbf{\$} & $\lessdot$ & $\lessdot$ & $\lessdot$ & $\lessdot$ & $\lessdot$ &  \\				
			\end{tabular}
		\end{center}
	\end{table}


\end{document}